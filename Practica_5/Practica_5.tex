\documentclass[a4paper]{article}

\usepackage[margin=2cm]{geometry}	% For smaller margins
\usepackage[catalan]{babel} 		% Catalan language 
\usepackage{fontspec}				% utf-8 support

\usepackage{amsmath}				% Math symbols
\usepackage{float}					% Better positioning 
\usepackage[hidelinks]{hyperref}	% To hide some links
\usepackage{pgfplots}				% Useful to plot data tables
\usepackage{enumitem}				% To use resume in enumerate
\usepackage{multirow}				% Multirow
\usepackage{gensymb} 				% For the degree symbol
\usepackage{graphicx}				% To include images
\usepackage{array}					% To alter column specification
\usepackage{pgfplotstable}			% To add CSV as tables

\pgfplotsset{compat=1.13}

\newenvironment{questionenum}{%
	\setlist[enumerate]{resume}
	\restartlist{enumerate}
	\newcommand{\question}[1]{
		\begin{enumerate}
			\item\bfseries ##1
		\end{enumerate}
}}{%
}

\setlength{\parindent}{0pt}
\setlength{\parskip}{1em}

\title{
	\textsc{Laboratori de Termodinàmica} \\
	\textsc{Pràctica 5} \\
	Aire humit \\
	\large
	Professor: Jose Luís \\ Grup: 11 }
\author{Joan Marcè Igual \and Esteve Tarragó Sanchís}
\date{13 de novembre de 2016}

\begin{document}
\maketitle

\section{Objectius}

La pràctica té per objectiu la determinació experimental de les propietats d'estat de l'aire humit i l'estudi d'alguns processos de condicionament d'aire com són l'escalfament i el refredament sensible. Es treballarà amb un equip de condicionament d'aire convencional amb compressor inverter i refrigerant R-410A accessible i instrumentat adequadament (dissenyat com equip didàctic). Els estats d'aire humit es determinaran amb un psicròmetre i els processos es visualitzaran en un diagrama psicromètric que també s'utilitzarà per a determinar la resta de paràmetres característics de l'aire: humitat relativa, humitat absoluta, volum específic, entalpia, etc. Finalment es realitzaran els balanços de matèria i energia dels processos estudiats.

\section{Presentació de resultats}

\begin{questionenum}
	\question{Presenteu una taula amb les propietats de l'aire del laboratori. Compareu els diferents sistemes de mesura emprats.}
	
	\question{Establiu les condicions termodinàmiques en què es troba el refrigerant dins de la unitat d'aire condicionat quan aquesta no es troba en funcionament.}
	
	\question{Presenteu dues taules (bateria interior i exterior) que continguin les temperatures, les humitats relatives i les velocitats determinades a diferents punts de la sortida de les bateries i els valors promig calculats.}
	
	\begin{table}[H]
		\centering
		\pgfplotstabletypeset[
			every head row/.style={ 
				after row=\hline	
			},
			create col/expr accum={\pgfmathaccuma+1}{0},
			display columns/0/.style={column name = T (ºC)},
			display columns/1/.style={column name = $\phi$ (\%)},
			display columns/2/.style={column name = $\vec{v}$ (m/s)},
			use comma,
			dec sep align
		]{data/unitat-interior.csv}
	\end{table}
\end{questionenum}

\end{document}