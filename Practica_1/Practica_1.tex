\documentclass[a4paper]{article}

\usepackage[catalan]{babel} % Language 
\usepackage{fontspec}
\usepackage[margin=2cm]{geometry}
\usepackage{float}
\usepackage{pgfplots}
\usepackage{pgfplotstable}
\usepackage[hidelinks]{hyperref}
\usepackage{amsmath}
\usepackage{enumitem}
\usepackage{scrextend}
\usepackage{csvsimple}
\usepackage{gensymb}
\usepackage{array}

\newcolumntype{L}[1]{>{\raggedright\let\newline\\\arraybackslash\hspace{0pt}}m{#1}}
\newcolumntype{C}[1]{>{\centering\let\newline\\\arraybackslash\hspace{0pt}}m{#1}}
\newcolumntype{R}[1]{>{\raggedleft\let\newline\\\arraybackslash\hspace{0pt}}m{#1}}

\usepgfplotslibrary{external} 
\tikzexternalize

\pgfplotsset{compat=1.14}

\setlength{\parindent}{0pt}
\setlength{\parskip}{1em}

\title{
	\textsc{Laboratori de Termodinàmica} \\
	\textsc{Pràctica 1} \\
	Propietats PVT d'una substància pura \\
	\large
	Professor: Jose Luís \\ Grup: 11 }
\author{Joan Marcè Igual \and Esteve Tarragó Sanchís}
\date{11 d'octubre de 2016}

\begin{document}
\maketitle

\section*{Objectius}
La pràctica té per objecte principal determinar experimentalment les relacions pressió, volum i temperatura ($PvT$) d'una substància pura i observar-ne el comportament en diferents situacions (líquid, vapor, equilibri líquid-vapor, punt crític i gas). Per fer això es determinaran diferents isotermes, la corba de saturació i les coordenades del punt crític del gas. Els resultats es presentaran en forma de diagrames $P-T$ i $P-v$.

Un segon objecte de la pràctica és comprendre el concepte d’equació d’estat i mostrar-ne la seva importància dins del càlcul de propietats termodinàmiques. Per fer això es buscarà quina equació d’estat es més adequada per descriure el comportament d’una substància en funció de la precisió i complexitat del càlcul.


\section*{Obtenció i presentació de resultats}

\begin{enumerate}
	\item \textbf{Anoteu els valors obtinguts en taules com aquestes o en d’altres de semblants (podeu fer servir el full de càlcul anomenat \textbf{Propietats $SF_6$ (full Experimental)} que utilitzareu en els propers apartats). Comenteu els fenòmens observats durant les experiències.}
\end{enumerate}

\begin{table}[H]
	\begin{minipage}[t][8,3cm][b]{0.49\linewidth}
		\begin{tabular}{rrr}
			\multicolumn{3}{c}{\bf Temperatura (K), T = 27ºC} \\
			Pressió [MPa] & Volum [$cm^3$] & v [$cm^3 · g^{-1}$] \\
			\hline
			\csvreader[
					late after line=\\, 
					head to column names, separator=semicolon
				]
				{experimental_subcritica.csv}{}{\P & \V & \v }
		\end{tabular}
		\vfill
		\caption{Isoterma subcrítica}
		\label{tab:subcritica}
	\end{minipage}
	\begin{minipage}[t][8,3cm][b]{0.49\linewidth}
		\begin{tabular}{rrr}
			\multicolumn{3}{c}{\bf Temperatura (K), T = 55ºC} \\
			Pressió [MPa] & Volum [$cm^3$] & v [$cm^3 · g^{-1}$] \\
			\hline
			\csvreader[
					late after line=\\, 
					head to column names, separator=semicolon
				]
				{experimental_supercritica.csv}{}{\P & \V & \v }
		\end{tabular}
		\vfill
		\caption{Isoterma supercrítica}
		\label{tab:supercritica}
	\end{minipage}
\end{table}

\begin{table}[H]
	\centering
	\begin{tabular}{rr}
		\multicolumn{2}{l}{\textbf{Volum [$cm^3$], $\boldsymbol{V} = 0,578\ cm^3$ }} \\ 
		\multicolumn{2}{l}{\textbf{Volum [$cm^3 · g^{-1}$], $\boldsymbol{v} = 1,351\ cm^3·g^{-1}$ }}  \\
		Pressió [MPa] & Temperatura [K] \\
		\hline
		\csvreader[
				late after line=\\, 
				head to column names, separator=semicolon
			]
			{experimental_isocorica.csv}{}{ \P & \T }
	\end{tabular}
	\caption{Taula Isocora}
	\label{tab:isocora}
\end{table}

A la \autoref{tab:subcritica} (subcrítica) es pot observar que a partir d'un volum de $0,9\ cm^3$ la pressió es manté constant a $2,55\ MPa$ i que torna a augmentar a partir de $1,7\ cm^3$. Així doncs en aquest interval de volum el $SF_6$ està passant de fase gasosa a líquida. En canvi, aquest fenomen no es pot veure a la \autoref{tab:supercritica} (supercrítica) ja que el $SF_6$ es troba per sobre de la seva temperatura crítica i no arriba a passar mai a fase líquida.

Pel que fa a la \autoref{tab:isocora} (isocora) durant l'experiència es va poder observar que al voltant de la $T_c\ (46 \degree C)$ desapareixia la línia que separa la fase líquida de la de vapor.

\begin{enumerate}[resume]
	\item \textbf{Els resultats de les experiències isotèrmiques es representen en un diagrama P-v, i es comparen amb les dades tabulades que trobareu a les taules dels annexes o en el full de càlcul anomenat \textbf{Propietats $\boldmath SF_6$ (DiagramaPv)}. Discutiu les diferencies observades entre les dades tabulades i les experimentals.}
\end{enumerate}

Al gràfic es poden observar les diferents isotermes tabulades i les dues experimentals realitzades a la pràctica. En totes dues isotermes hi ha certs punts destacats de la resta que són deguts a errors en la mesura.

Pel que fa a la isoterma subcrítica es pot veure com hi ha tres parts diferenciades. La primera part és la compressió en estat gasos; la segona, és el procés de liquació on la pressió es manté gairebé constant i la tercera és el procés de compressió en estat líquid. Cal destacar que s'observa com el procés de liquació sembla començar abans del valor teòric. També es pot veure com la tercera part del procés la pressió ha augmentat molt ràpidament respecte la pressió en la segona part.

Pel que fa a la isoterma supercrítica no s'observa cap tram diferenciat i durant l'experiència tampoc es va observar cap canvi d'estat. 

\begin{figure}[H]
    \centering
	\begin{tikzpicture}
		\begin{semilogxaxis}[
                xlabel = v ($m^3 · kg^{-1}$),
                ylabel = P (MPa),
                xmin = 0.0004, xmax = 0.1,
                ymin = 0, ymax = 5.5,
                xtick = {0.0001, 0.001, 0.01, 0.1},
                log ticks with fixed point,
                width=\textwidth, height=0.6\textwidth,
                legend style={draw=none},
                legend cell align = left,
                cycle list name = color list,
            ]
            \addplot[black, style = {line width=1.6pt}] table[mark=none, x=vl, y=P, col sep = semicolon, forget plot]{saturacio.csv};
            \addplot[black, style = {line width=1.6pt}] table[mark=none, x=vv, y=P, col sep = semicolon, forget plot]{saturacio.csv};
            \addplot+[mark=*, only marks] table[x=v2, y=P2, col sep = semicolon]{experimental_supercritica.csv};
            \addplot+[mark=*, only marks] table[x=v2, y=P2, col sep = semicolon]{experimental_subcritica.csv};
            \addplot+[mark=none] table[x=v-67, y=P-67, col sep = semicolon]{isotermes.csv};
            \addplot+[mark=none] table[x=v-60, y=P-60, col sep = semicolon]{isotermes.csv};
            \addplot+[mark=none] table[x=v-55, y=P-55, col sep = semicolon]{isotermes.csv};
            \addplot+[purple, mark=none] table[x=v-51, y=P-51, col sep = semicolon]{isotermes.csv};
            \addplot+[mark=none] table[x=v-45.58, y=P-45.58, col sep = semicolon]{isotermes.csv};
            \addplot+[mark=none] table[x=v-40, y=P-40, col sep = semicolon]{isotermes.csv};
            \addplot+[mark=none] table[x=v-35, y=P-35, col sep = semicolon]{isotermes.csv};
            \addplot+[mark=none] table[x=v-27, y=P-27, col sep = semicolon]{isotermes.csv};
            \addplot+[mark=none] table[x=v-17, y=P-17, col sep = semicolon]{isotermes.csv};
            \addplot+[mark=none] table[x=v-7, y=P-7, col sep = semicolon]{isotermes.csv};
            \legend{Supercrítica, Subcrítica, T=67ºC, T=60ºC, T=55ºC, T=51ºC, T=45.48ºC, T=40ºC, T=35ºC, T=27ºC, T=17ºC,T=7ºC}
        \end{semilogxaxis}
	\end{tikzpicture}
\end{figure}

\begin{enumerate}[resume]
    \item \textbf{Els resultats de l’experiència isocòrica es representen en un diagrama $P-T$ i es comparen amb les dades tabulades que trobareu a les taules dels annexes o en el full de càlcul anomenat \textbf{Propietats $SF_6$ (DiagramaPT)}. El procés isocòric si es vol es pot representar també en el diagrama $P-v$. Compareu també el punt crític experimental amb el teòric.}
\end{enumerate}

\begin{figure}[H]
    \centering
    \begin{tikzpicture}
        \begin{axis}[
                xlabel = T(K),
                ylabel = P(MPa),
                xmin = 220, xmax = 340,
                ymin = 0, ymax = 4.5,
                width = \textwidth, height = 0.6\textwidth,
                legend style={draw=none},
                legend cell align = left,
            ]
            \addplot[mark=none] table [x=T, y=P, col sep = semicolon]{tabulades_pt.csv};
            \addplot[mark=*] table [x=T, y=Pexp, col sep = semicolon]{experimental_pt.csv};
        \end{axis}
    \end{tikzpicture}
\end{figure}

El punt crític teòric del $SF_6$ és de $P_c = 3,759 MPa$ i $T_c = 45,58\degree C$ i el punt crític obtingut a la pràctica ha estat de $T_c = 46 \degree C$ i $P_c = 3,9MPa$.

\begin{addmargin}[3em]{0em}
    \textbf{Si bé, en una primera aproximació, la corba de vaporització es pot descriure mitjançant l’equació de Clausius-Clapeyron, quan es treballa a pressions elevades, es requereixen equacions més complexes com pot ésser l’obtinguda per Funke \textit{et al.(J. Chem Thermodynamics,} 2001, 34, 735-754) pel SF6: }
    
    $$
    ln\left(\frac{P}{P_c}\right) = \left(\frac{T}{T_c}\right)
    \left[ -7,12 \left( 1 - \frac{T}{T_c} \right) + 2,04 \left( 1 - \frac{T}{T_c} \right)^{1,5}
    - 1,55 \left( 1 - \frac{T}{T_c} \right)^{2} - 2,64 \left( 1 - \frac{T}{T_c} \right)^4 \right]
    $$
    
    \textbf{Utilitzant l’equació anterior, per les dades del procés isocòric, calculeu l’error comès en la determinació experimental de la pressió de saturació. Discutiu el resultat obtingut. Aquesta equació la trobareu en el full de càlcul anomenat Propietats$SF_6$ (Psaturació).}
\end{addmargin}


\begin{table}[H]
    \centering
    \begin{tabular}{rrrr}
        $T_{exp} (K)$ & $P_{exp} (MPa)$ & $P_{teorica} (MPa)$ & \% error \\
        \hline
        \csvreader[
                late after line=\\, 
                head to column names, separator=semicolon
            ]
            {error_pt.csv}{}{ \T & \Pexp & \Pt & \err }
    \end{tabular}
    \caption{Taula Isocora amb error}
    \label{tab:isocora_err}
\end{table}

\begin{enumerate}[resume]
    \item \textbf{També es pot treballar el concepte de \emph{títol} i comprovar que el volum d’estats bifàsics es pot descriure a través del títol i dels volums de vapor i líquid saturat. Es calcularà el títol inicial del procés isocòric utilitzant el volum experimental i les dades tabulades dels volums de vapor i líquid saturat.}
\end{enumerate}
Les condicions inicials del proces isocoric són:
		
	$\boldsymbol{V} = 0,578\ cm^3$	
	$\boldsymbol{m} = 0,42813 g$	
	$\boldsymbol{v} = 1,351\ cm^3·g^{-1}$	
	$\boldsymbol{T} = 300 ºK$	
	$\boldsymbol{P} = 2,65 MPa$
	
	Consultant a les taules s'obté:
	$\boldsymbol{v_v} = 4,194\ cm^3·g^{-1} $
	$\boldsymbol{v_l} = 0,758\ cm^3·g^{-1} $
	
	Resolent la següent equació s'obté el titol de estat inicial:
	
	$\boldsymbol{v} = \boldsymbol{x}* \boldsymbol{v_v} + (1- \boldsymbol{x}) * \boldsymbol{v_l}$
	
	$\boldsymbol{x} = 0,1726 $
	
	

\begin{enumerate}[resume]
    \item \textbf{Per determinar quina equació d’estat reprodueix millor les propietats del $SF_6$ es calcularà l’error que tindria lloc si el gas seguís diferents equacions d’estat. Per fer aquest càlculs cal utilitzar el full de càlcul anomenat \textbf{ErrorsEquacions}. En aquest full, per unes quantes dades (2 o 3) de les dues isotermes experimentals (una subcrítica i una supercrítica), es calcularà l’error comparant les pressions experimentals amb les determinades amb les diferents equacions d’estat. Seleccioneu l’equació que doni menys error per resoldre el següent apartat.}
\end{enumerate}


\begin{table}[H]
	\centering
	\begin{tabular}{@{}R{1.5cm}R{2cm}R{2cm}R{1cm}R{1.3cm}R{1.3cm}R{1cm}R{1.5cm}R{1.3cm}@{} }
		Pressió [MPa] & volum específic [$cm^3 · g^{-1}$] & Temperatura [K] & Gas Ideal & Van der Waals & Redlich Kwong & Soave & Peng Robinson & BWR Han Starling \\
		\hline
		\csvreader[
        		late after line=\\, 
        		head to column names, separator=semicolon
    		]
    		{problema5.csv}{}{\P & \v & \T & \GasIdeal & \VanderWaals & \RedlichKwong & \Soave & \PengRobinson & \BWRHanStarling }
        \hline
        \multicolumn{3}{c}{Mitjana error absolut} & 107,88 & 178,40 & 17,98 & 13,20 & 11,82 & 15,90
	\end{tabular}
	\caption{Taula d'errors d'equacions}
	\label{tab:errors}
\end{table}
Per construir la taula, s'han decidit agafar els dos punts extrems i el central de cada una de les isotermes ja que s'ha considerat que aixi era més representatiu de tot el procés.
Com podem s'observa a la taula, l'equació amb menys error porcentual de mitjana és la de Peng Robinson amb un valor mig del 11,82 \%.

\begin{enumerate}[resume]
    \item \textbf{Les dues isotermes experimentals es compararan amb les obtingudes utilitzant l’equació de gas ideal i l’equació que hagi donat un error més petit a l’apartat anterior. Aquesta comparació la podeu fer en un diagrama $Pv$ utilitzant el full de càlcul anomenat \textbf{Equacions}. Primer cal introduir les dades experimentals en \textbf{DadesGas} i el full de càlcul calcularà la isoterma utilitzant diferents equacions d’estat inclosa la de gas ideal. Finalment es pot graficar en \textbf{GraficEquació} les dades experimentals i les obtingudes mitjançant l’equació d’estat escollida i la de gas ideal. Discutiu el resultat obtingut, especialment en quan a la bondat dels diferents mètode d’estimació de propietats i a les formes que presenten les isotermes.}
\end{enumerate}

\begin{figure}[H]
    \centering
    \begin{tikzpicture}
        \begin{axis}[
                xlabel = v($cm^3·g^{-1}$),
                ylabel = P (MPa),
                xmin = 0, xmax = 9,
                ymin = 1.5, ymax = 5.5,
                width=\textwidth, height=0.8\textwidth,
                legend style={draw=none},
                legend cell align = left,
                cycle list name = exotic,
            ]
            \addplot+[mark=*, only marks] table[x=v3, y=P2, col sep = semicolon]{experimental_supercritica.csv};
            \addplot+[mark=*, only marks] table[x=v3, y=P2, col sep = semicolon]{experimental_subcritica.csv};
            \addplot+[smooth, mark=none] table[x=v,y=P,col sep = semicolon]{gas_ideal.csv};
            \addplot+[smooth, mark=none] table[x=v,y=P,col sep = semicolon]{bwr.csv};
            \addplot+[smooth, mark=none] table[x=v,y=P,col sep = semicolon]{peng_robinson.csv};
            \addplot+[smooth, mark=none] table[x=v,y=P,col sep = semicolon]{redlich_kwong.csv};
            \addplot+[smooth, mark=none] table[x=v,y=P,col sep = semicolon]{soave.csv};
            \addplot+[smooth, mark=none] table[x=v,y=P,col sep = semicolon]{van_der_waals.csv};
            \legend{Experimental supercrítica, Experimental subcrítica, Gas Ideal, Benedict Webb Rubin, Peng i Robinson, Redlich Kwong, Soave, Van der Waals};
        \end{axis}
    \end{tikzpicture}
\end{figure}

Imatge 27.jpg

Es pot observar que la corba queda per sota en la majoria de casos de les dades obtingudes experimentalment. Això pot ser causat per una temperatura lleugerament superior durant l'experimentació i errors de mesura. Cal destacar que l'equacio de Peng i Robinson s'ajusta a les dades en la fase liquida, en la de vapor i en la del gas. Tot hi així cal recordar que com totes les equacions teoriques només s'aproximen a la realitat.

En contraposició la equació del gas ideal es va distancian de les dades experimentals a mesura que aumenta la presió, com era de suposar ja que no te en compte interaccions entre les molecules que augmentan a mesura que augmenta la presió.



Imatge 55.jpg


Com en la gràfica anterior les dades experimentals queden per sobre la corba de presió-volum de l'equacio de Peng i Robinson. També s'observa com l'equació del gas ideal obté menys error en presions baixes que la equacio de Peng i Robinson i, a mesura que aumenta la presió, les dades es distancien de la corba del gas ideal i s'aproximen a les de l'equacio de Peng i Robinson.

\end{document}