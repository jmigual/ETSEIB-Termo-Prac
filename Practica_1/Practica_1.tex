\documentclass[a4paper]{article}

\usepackage[catalan]{babel} % Language 
\usepackage{fontspec}
\usepackage[margin=2cm]{geometry}
\usepackage{float}
\usepackage{pgfplots}
\usepackage{pgfplotstable}
\usepackage[hidelinks]{hyperref}
\usepackage{amsmath}
\usepackage{enumitem}
\usepackage{csvsimple}

\pgfplotsset{compat=1.14}

\setlength{\parindent}{0pt}
\setlength{\parskip}{1em}

\title{
	\textsc{Laboratori de Termodinàmica} \\
	\textsc{Pràctica 1} \\
	Propietats PVT d'una substància pura \\
	\large
	Professor: Jose Luís \\ Grup: 11 }
\author{Joan Marcè Igual \and Esteve Tarragó Sanchís}
\date{11 d'octubre de 2016}

\begin{document}
\maketitle

\section*{Objectius}
La pràctica té per objecte principal determinar experimentalment les relacions pressió, volum i temperatura ($PvT$) d'una substància pura i observar-ne el comportament en diferents situacions (líquid, vapor, equilibri líquid-vapor, punt crític i gas). Per fer això es determinaran diferents isotermes, la corba de saturació i les coordenades del punt crític del gas. Els resultats es presentaran en forma de diagrames $P-T$ i $P-v$.

Un segon objecte de la pràctica és comprendre el concepte d’equació d’estat i mostrar-ne la seva importància dins del càlcul de propietats termodinàmiques. Per fer això es buscarà quina equació d’estat es més adequada per descriure el comportament d’una substància en funció de la precisió i complexitat del càlcul.


\section*{Obtenció i presentació de resultats}

\begin{enumerate}
	\item Anoteu els valors obtinguts en taules com aquestes o en d’altres de semblants (podeu fer servir el full de càlcul anomenat \textbf{Propietats $\boldmath SF_6$ (full Experimental)} que utilitzareu en els propers apartats). Comenteu els fenòmens observats durant les experiències.
\end{enumerate}

\begin{table}[H]
	\begin{minipage}{0.49\linewidth}
		\begin{tabular}{rrr}
			\multicolumn{3}{c}{\bf Temperatura (K), T = } \\
			\hline
			Pressió [MPa] & Volum [$cm^3$] & v [$cm^3 · g^{-1}$]
		\end{tabular}
		\caption{Taula Isoterma subcrítica}
	\end{minipage}
	\begin{minipage}{0.49\linewidth}
		\begin{tabular}{rrr}
			\multicolumn{3}{c}{\bf Temperatura (K), T = } \\
			\hline
			Pressió [MPa] & Volum [$cm^3$] & v [$cm^3 · g^{-1}$]
			
		\end{tabular}
		\caption{Taula Isoterma supercrítica}
	\end{minipage}
\end{table}

\begin{table}[H]
	\centering
	\begin{tabular}{rr}
		\textbf{Volum [$cm^3$], V = } & \textbf{Volum [$cm^3 · g^{-1}$], v = } \\
		\hline
		Pressió [MPa] & Temperatura [K] \\
	\end{tabular}
	\caption{Taula Isocora}
\end{table}

\begin{enumerate}[resume]
	\item Els resultats de les experiències isotèrmiques es representen en un diagrama P-v, i es comparen amb les dades tabulades que trobareu a les taules dels annexes o en el full de càlcul anomenat \textbf{Propietats $\boldmath SF_6$ (DiagramaPv)}. Discutiu les diferencies observades entre les dades tabulades i les experimentals.
\end{enumerate}

\begin{figure}[H]
	\begin{tikzpicture}
		
	\end{tikzpicture}
\end{figure}

\end{document}