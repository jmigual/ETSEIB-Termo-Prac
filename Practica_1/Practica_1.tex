\documentclass[a4paper]{article}

\usepackage[catalan]{babel} % Language 
\usepackage{fontspec}
\usepackage[margin=2cm]{geometry}
\usepackage{float}
\usepackage{pgfplots}
\usepackage{pgfplotstable}
\usepackage[hidelinks]{hyperref}
\usepackage{amsmath}
\usepackage{enumitem}
\usepackage{csvsimple}
\usepackage{gensymb}

\usepgfplotslibrary{external} 
\tikzexternalize

\pgfplotsset{compat=1.14}

\setlength{\parindent}{0pt}
\setlength{\parskip}{1em}

\title{
	\textsc{Laboratori de Termodinàmica} \\
	\textsc{Pràctica 1} \\
	Propietats PVT d'una substància pura \\
	\large
	Professor: Jose Luís \\ Grup: 11 }
\author{Joan Marcè Igual \and Esteve Tarragó Sanchís}
\date{11 d'octubre de 2016}

\begin{document}
\maketitle

\section*{Objectius}
La pràctica té per objecte principal determinar experimentalment les relacions pressió, volum i temperatura ($PvT$) d'una substància pura i observar-ne el comportament en diferents situacions (líquid, vapor, equilibri líquid-vapor, punt crític i gas). Per fer això es determinaran diferents isotermes, la corba de saturació i les coordenades del punt crític del gas. Els resultats es presentaran en forma de diagrames $P-T$ i $P-v$.

Un segon objecte de la pràctica és comprendre el concepte d’equació d’estat i mostrar-ne la seva importància dins del càlcul de propietats termodinàmiques. Per fer això es buscarà quina equació d’estat es més adequada per descriure el comportament d’una substància en funció de la precisió i complexitat del càlcul.


\section*{Obtenció i presentació de resultats}

\begin{enumerate}
	\item Anoteu els valors obtinguts en taules com aquestes o en d’altres de semblants (podeu fer servir el full de càlcul anomenat \textbf{Propietats $\boldmath SF_6$ (full Experimental)} que utilitzareu en els propers apartats). Comenteu els fenòmens observats durant les experiències.
\end{enumerate}

\begin{table}[H]
	\begin{minipage}[t][8,3cm][b]{0.49\linewidth}
		\begin{tabular}{rrr}
			\multicolumn{3}{c}{\bf Temperatura (K), T = 27ºC} \\
			Pressió [MPa] & Volum [$cm^3$] & v [$cm^3 · g^{-1}$] \\
			\hline
			\csvreader[
					late after line=\\, 
					head to column names, separator=semicolon
				]
				{experimental_subcritica.csv}{}{\P & \V & \v }
		\end{tabular}
		\vfill
		\caption{Isoterma subcrítica}
		\label{tab:subcritica}
	\end{minipage}
	\begin{minipage}[t][8,3cm][b]{0.49\linewidth}
		\begin{tabular}{rrr}
			\multicolumn{3}{c}{\bf Temperatura (K), T = 55ºC} \\
			Pressió [MPa] & Volum [$cm^3$] & v [$cm^3 · g^{-1}$] \\
			\hline
			\csvreader[
					late after line=\\, 
					head to column names, separator=semicolon
				]
				{experimental_supercritica.csv}{}{\P & \V & \v }
		\end{tabular}
		\vfill
		\caption{Isoterma supercrítica}
		\label{tab:supercritica}
	\end{minipage}
\end{table}

\begin{table}[H]
	\centering
	\begin{tabular}{rr}
		\multicolumn{2}{l}{\textbf{Volum [$cm^3$], $\boldsymbol{V} = 0,578\ cm^3$ }} \\ 
		\multicolumn{2}{l}{\textbf{Volum [$cm^3 · g^{-1}$], $\boldsymbol{v} = 1,351\ cm^3·g^{-1}$ }}  \\
		Pressió [MPa] & Temperatura [K] \\
		\hline
		\csvreader[
				late after line=\\, 
				head to column names, separator=semicolon
			]
			{experimental_isocorica.csv}{}{ \P & \T }
	\end{tabular}
	\caption{Taula Isocora}
	\label{tab:isocora}
\end{table}

A la \autoref{tab:subcritica} (subcrítica) es pot observar que a partir d'un volum de $0,9\ cm^3$ la pressió es manté constant a $2,55\ MPa$ i que torna a augmentar a partir de $1,7\ cm^3$. Així doncs en aquest interval de volum el $SF_6$ està passant de fase gasosa a líquida. En canvi, aquest fenomen no es pot veure a la \autoref{tab:supercritica} (supercrítica) ja que el $SF_6$ es troba per sobre de la seva temperatura crítica i no arriba a passar mai a fase líquida.

Pel que fa a la \autoref{tab:isocora} (isocora) durant l'experiència es va poder observar que al voltant de la $T_c\ (46 \degree C)$ desapareixia la línia que separa la fase líquida de la de vapor.

\begin{enumerate}[resume]
	\item Els resultats de les experiències isotèrmiques es representen en un diagrama P-v, i es comparen amb les dades tabulades que trobareu a les taules dels annexes o en el full de càlcul anomenat \textbf{Propietats $\boldmath SF_6$ (DiagramaPv)}. Discutiu les diferencies observades entre les dades tabulades i les experimentals.
\end{enumerate}

\begin{figure}[H]
    \centering
	\begin{tikzpicture}
		\begin{semilogxaxis}[
                xlabel = v ($m^3 · kg^{-1}$),
                ylabel = P (MPa),
                xmin = 0.0004, xmax = 0.1,
                ymin = 0, ymax = 6,
                xtick = {0.0001, 0.001, 0.01, 0.1},
                log ticks with fixed point,
                width=\textwidth, height=0.8\textwidth,
                legend style={draw=none},
                legend cell align = left,
                cycle list name = color list,
            ]
            \addplot[black, style = {line width=1.6pt}] table[mark=none, x=vl, y=P, col sep = semicolon, forget plot]{saturacio.csv};
            \addplot[black, style = {line width=1.6pt}] table[mark=none, x=vv, y=P, col sep = semicolon, forget plot]{saturacio.csv};
            \addplot+[mark=*, only marks] table[x=v2, y=P2, col sep = semicolon]{experimental_supercritica.csv};
            \addplot+[mark=*, only marks] table[x=v2, y=P2, col sep = semicolon]{experimental_subcritica.csv};
            \addplot+[mark=none] table[x=v-67, y=P-67, col sep = semicolon]{isotermes.csv};
            \addplot+[mark=none] table[x=v-60, y=P-60, col sep = semicolon]{isotermes.csv};
            \addplot+[mark=none] table[x=v-55, y=P-55, col sep = semicolon]{isotermes.csv};
            \addplot+[purple, mark=none] table[x=v-51, y=P-51, col sep = semicolon]{isotermes.csv};
            \addplot+[mark=none] table[x=v-45.58, y=P-45.58, col sep = semicolon]{isotermes.csv};
            \addplot+[mark=none] table[x=v-40, y=P-40, col sep = semicolon]{isotermes.csv};
            \addplot+[mark=none] table[x=v-35, y=P-35, col sep = semicolon]{isotermes.csv};
            \addplot+[mark=none] table[x=v-27, y=P-27, col sep = semicolon]{isotermes.csv};
            \addplot+[mark=none] table[x=v-17, y=P-17, col sep = semicolon]{isotermes.csv};
            \addplot+[mark=none] table[x=v-7, y=P-7, col sep = semicolon]{isotermes.csv};
            \legend{Supercrítica, Subcrítica, T=67ºC, T=60ºC, T=55ºC, T=51ºC, T=45.48ºC, T=40ºC, T=35ºC, T=27ºC, T=17ºC,T=7ºC}
        \end{semilogxaxis}
	\end{tikzpicture}
\end{figure}

\begin{enumerate}[resume]
    \item Els resultats de l’experiència isocòrica es representen en un diagrama $P-T$ i es comparen amb les dades tabulades que trobareu a les taules dels annexes o en el full de càlcul anomenat \textbf{Propietats $SF_6$ (DiagramaPT)}. El procés isocòric si es vol es pot representar també en el diagrama $P-v$. Compareu també el punt crític experimental amb el teòric.
\end{enumerate}

\begin{enumerate}[resume]
    \item També es pot treballar el concepte de \emph{títol} i comprovar que el volum d’estats bifàsics es pot descriure a través del títol i dels volums de vapor i líquid saturat. Es calcularà el títol inicial del procés isocòric utilitzant el volum experimental i les dades tabulades dels volums de vapor i líquid saturat.
\end{enumerate}
Les condicions inicials del proces isocoric són:
		
	$\boldsymbol{V} = 0,578\ cm^3$	
	$\boldsymbol{m} = 0,42813 g$	
	$\boldsymbol{v} = 1,351\ cm^3·g^{-1}$	
	$\boldsymbol{T} = 300 ºK$	
	$\boldsymbol{P} = 2,65 MPa$
	
	Consultant a les taules s'obté:
	$\boldsymbol{v_v} =  $
	$\boldsymbol{v_l} = $
	
	Resolent la següent equació s'obté el titol de estat inicial:
	
	$\boldsymbol{v} = \boldsymbol{x}* \boldsymbol{v_v} + (1- \boldsymbol{x}) * \boldsymbol{v_l}$
	
	$\boldsymbol{x} = $
	
	

\begin{enumerate}[resume]
    \item Per determinar quina equació d’estat reprodueix millor les propietats del $SF_6$ es calcularà l’error que tindria lloc si el gas seguís diferents equacions d’estat. Per fer aquest càlculs cal utilitzar el full de càlcul anomenat \textbf{ErrorsEquacions}. En aquest full, per unes quantes dades (2 o 3) de les dues isotermes experimentals (una subcrítica i una supercrítica), es calcularà l’error comparant les pressions experimentals amb les determinades amb les diferents equacions d’estat. Seleccioneu l’equació que doni menys error per resoldre el següent apartat.
\end{enumerate}


\begin{table}[H]
	\centering
	\begin{tabular}{rr}
		\multicolumn{11}{l}{\textbf{Titol o algo aixi?}}  \\
		Resultats & Pressió [MPa] & Volum[$cm^3$] & volum específic [$cm^3 · g^{-1}$] & Temperatura [K] & Gas Ideal & VanderWaals & Redlich Kwong & Soave & Peng Robinson & BWR HanStarling \\
		\hline
		\csvreader[
		late after line=\\, 
		head to column names, separator=semicolon
		]
		{problema5.csv}{}{\Resultats & \P & \V & \v & \T & \GasIdeal & \VanderWaals & \RedlichKwong & \Soave & \PengRobinson & \BWRHanStarling }
	\end{tabular}
	\caption{Taula d'errors d'equacions}
	\label{tab:errors}
\end{table}

\begin{enumerate}[resume]
    \item Les dues isotermes experimentals es compararan amb les obtingudes utilitzant l’equació de gas ideal i l’equació que hagi donat un error més petit a l’apartat anterior. Aquesta comparació la podeu fer en un diagrama $Pv$ utilitzant el full de càlcul anomenat \textbf{Equacions}. Primer cal introduir les dades experimentals en \textbf{DadesGas} i el full de càlcul calcularà la isoterma utilitzant diferents equacions d’estat inclosa la de gas ideal. Finalment es pot graficar en \textbf{GraficEquació} les dades experimentals i les obtingudes mitjançant l’equació d’estat escollida i la de gas ideal. Discutiu el resultat obtingut, especialment en quan a la bondat dels diferents mètode d’estimació de propietats i a les formes que presenten les isotermes.
\end{enumerate}

\begin{figure}[H]
    \centering
    \begin{tikzpicture}
        \begin{axis}[
                xlabel = v($cm^3·g^{-1}$),
                ylabel = P (MPa),
                xmin = 0, xmax = 9,
                ymin = 1.5, ymax = 5.5,
                width=\textwidth, height=0.8\textwidth,
                legend style={draw=none},
                legend cell align = left,
                cycle list name = exotic,
            ]
            \addplot+[smooth, mark=none] table[x=v,y=P,col sep = semicolon]{gas_ideal.csv};
            \addplot+[smooth, mark=none] table[x=v,y=P,col sep = semicolon]{bwr.csv};
            \addplot+[smooth, mark=none] table[x=v,y=P,col sep = semicolon]{peng_robinson.csv};
            \addplot+[smooth, mark=none] table[x=v,y=P,col sep = semicolon]{redlich_kwong.csv};
            \addplot+[smooth, mark=none] table[x=v,y=P,col sep = semicolon]{soave.csv};
            \addplot+[smooth, mark=none] table[x=v,y=P,col sep = semicolon]{van_der_waals.csv};
            \legend{Gas Ideal, Benedict Webb Rubin, Peng i Robinson, Redlich Kwong, Soave, Van der Waals};
        \end{axis}
    \end{tikzpicture}
\end{figure}

\end{document}