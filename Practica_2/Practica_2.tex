\documentclass[a4paper]{article}

\usepackage[catalan]{babel} % Language 
\usepackage{fontspec}
\usepackage[margin=2cm]{geometry}
\usepackage{pgfplots}
\usepackage{pgfplotstable}
\usepackage[hidelinks]{hyperref}
\usepackage{float}

\setlength{\parindent}{0pt}
\setlength{\parskip}{1em}

\title{\textsc{Laboratori de Termodinàmica} \\ \textsc{Pràctica 2} \\ Compressió de gasos a baixes pressions \\
    \large
    Professor: Jose Luís \\ Grup: 11 }
\author{Joan Marcè Igual \and Esteve Tarragó Sanchís}
\date{26-9-2016}

\begin{document}
\maketitle

\section*{Objectius}

L'objectiu d'aquesta pràctica és estudiar el comportament d'un gas ideal en diferents processos de compressió. Per aquests processos es determinarà el coeficient politròpic $n$ i, amb l'ajuda del balanç d'energia el treball i la calor intercanviats, es comprovarà experimentalment l'equació d'estat tèrmica d'un gas ideal. 

També, s'estudiarà el cicle termodinàmic, denominat d'Otto invers, que funcionant amb un gas ideal permet actuar com a bomba de calor o màquina frigorífica. S'analitzarà també una modificació del cicle d'Otto invers on l'etapa de compressió es realitza isotèrmicament.

S'utilitzarà un cilindre-pistó transparent amb un braç per poder realitzar les compressions i expansions dels gasos continguts a l'interior. El dispositiu té tres sensors que permeten la mesura simultània de pressió, volum i temperatura.

\section*{Obtenció i presentació de resultats}

\subsection*{Processos politròpics}

\textbf{1. Presenteu una taula que contingui la informació experimental i calculada següent:}

S'han realitzat dos assajos un en un procés adiabàtic (prova 1) i l'altre en un procés isotèrmic (prova 2). Per a cada assaig s'han obtingut les dades i també s'ha obtingut la recta d'interpolació d'aquestes dades ($ax^b$), a partir d'aquest exponent $b$ s'ha obtingut el coeficient politròpic (on $n = -b$).
\begin{center}
\begin{tabular}{c|ccccccccc}
    Prova & $V_1$ $(cm^3)$ & $T_1$ $(K)$ & $P_1$ $(kPA)$ & $V_2$ $(cm^3)$ & $T_2$ $(K)$ & $P_2$ $(kPA)$ & $n$ & $W(J)$ & $Q(J)$ \\
    \hline
    1 & 198,53 & 300 & 103,258 & 90,336 & 381 & 381,604 & 1,29 & 18,133 & -4,9866 \\
    2 & 198,692 & 300 & 102,719 & 90,498 & 302 & 228,262 & 1 & 16.051 & -16.051
\end{tabular}
\end{center}

Com es pot veure a la taula la pressió durant la prova 1 és superior a la pressió durant la prova 2 això és degut perquè a la prova 1 la temperatura també és superior (s'ha realitzat un procès pràcticament adiabàtic). 

Si es comparen les energies es pot observar que el treball ($W$) és superior en el cas adiabàtic tot i que és força semblant al cas isotèrmic. Tot i així, la calor ($Q$) és molt diferent. Això és degut perquè el procés isotèrmic ha anat alliberant en forma de calor tot el treball obtingut (s'ha suposat que $\Delta U \simeq 0$) en canvi el procés adiabàtic ha absorbit el treball aportat i només ha alliberat en forma de calor una petita part del treball aportat. Idealment la calor de la prova 1 però el recipient no és completament adiabàtic.

\textbf{2. Compareu el valor del coeficient politròpic de l’experiència de compressió més ràpida, amb el coeficient isentròpic de l’aire (k=1,4). Discutiu el valor obtingut i les possibles diferències entre els coeficients experimental i teòric.}

Com es pot veure a la taula anterior, el valor obtingut ($1.29$) ha estat inferior al valor esperat ($1.4$). Això pot ser degut a diferents factors.

El principal factor que pot haver provocat aquest fet és que és impossible fer una compressió totalment adiabàtica ja que sempre hi haurà pèrdues de calor per les parets del recipient i això provocarà una disminució de la temperatura no desitjada. També, ja que hi ha pèrdua de calor per les parets, si la compressió no es realitza prou ràpid aquest fenomen s'accentua.

\textbf{3. En un diagrama pressió-volum, representeu tots el processos de compressió realitzats. Relacioneu les àrees existents sota les gràfiques amb els valors dels treballs de les diferents experiències.}

En el següent gràfic (Pressió - Volum) s'han representat les dades obtingudes en els dos assajos anteriors i també s'hi ha afegit la funció d'interpolació usada. Es pot observar que la corba adiabàtica es troba per sobre de la isotèrmica segons la hipòtesi inicial, ja que els processos isotèrmics tenen $n=1$ i els adiabàtics $n=1.4$. 

Pel que fa a l'àrea sota de les corbes, aquestes són directament el treball realitzat. Es pot observar que l'àrea sota la corba del procès adiabàtic és superior a la del procés isotèrmic, de la mateixa manera que el treball del procés adiabàtic és superior al del isotèrmic.

\begin{figure}[!h]
\centering
\caption{Assaig adiabàtic  i isotèrmic}
\label{fig:asssaig_ad_iso}
\begin{tikzpicture}[scale=1]
\begin{axis}[
    xlabel = Volum ($cm^3$),
    ylabel = Pressió ($kPa$),
    ymajorgrids=true]
\addplot[
    only marks, 
    green, 
    mark=halfcircle*, 
    mark size=1pt, 
    opacity=0.5] 
    table [y=P, x=V, col sep=semicolon]{politropics_adiabatic.csv};
\addlegendentry{Procés adiabàtic};
\addplot[green, domain = 90:200] {96988*x^-1.292};
\addlegendentry{$96988x^{-1.29}$};
\addplot[
    only marks,
    red,  
    mark=halfcircle*, 
    mark size=1pt, 
    opacity=0.5] 
    table [y=P, x=V, col sep=semicolon]{politropics_isotermic.csv};
\addlegendentry{Procés isotèrmic};
\addplot[red, domain = 90:200] {20748*x^-1};
\addlegendentry{$20748x^{-1}$};
\end{axis}
\end{tikzpicture}
\end{figure}

\textbf{4. A l’experiència isotèrmica, podeu comprovar si el gas es comporta com ideal a l’interval de pressions empíric, determinant el producte PV. Aquest producte ha de ser constant si ho és la temperatura i el nombre de mols contingut al cilindre.}

Com es pot veure a la \autoref{fig:histograma_PV} les dades es troben distribuïdes segons una distribució normal amb uns paràmetres $\mu = 20762,86$ i $\sigma = 65.156$. Donat que la desviació estàndard de la mostra no és molt alta podem considerar que el fet que els valors no siguin tots iguals pot ser degut a l'error causat per l'aparell de mesura o pel fet de no ser un sistema ideal sinó que és un sistema real amb algunes pèrdues.

\begin{figure}[!h]
\centering
\caption{Histograma de PV}
\label{fig:histograma_PV}
\begin{tikzpicture}
\begin{axis}[
    ylabel = Freqüències dels valors,
    xlabel = PV,
    ybar,
    ymin = 0]
\addplot [
    hist={bins = 7,
        data min = 20409.44355,
        data max = 20907.12019 }] 
    table [y index = 0]{politropics_isotermic_PV.csv};
\end{axis}
\end{tikzpicture}
\end{figure}

\subsection*{Cicle d'Otto invers}

\textbf{5. Amb les dades experimentals completeu la següent taula:}

\begin{tabular}{c|cccc}
    & Estat 1 & Estat 2 & Estat 3 & Estat 4 \\
    \hline
    $P(kPa)$ & 154,983 & 60,692 & 68,774 & 184,618 \\
    $V(cm^3)$ & 91,853 & 198,527 & 198,527 & 91,371 \\
    $T(K)$ & 301,569º & 268,596 & 298,763 & 365,411 \\
\end{tabular}

\textbf{6. Determineu els treballs i les calors intercanviats a les quatre etapes del cicle utilitzant les equacions (7), (8), (9) i 10. Determineu els coeficients de funcionament del cicle com a màquina frigorífica i com a bomba de calor utilitzant les calors i les equacions (6). Discutiu els resultats obtinguts.}

\begin{tabular}{c|cccccc}
    Cicle & $Q_H(J)$ & $Q_L(J)$ & $W_{exp}(J)$ & $W_{comp}(J)$ & $COP_{MF}$ & $COP_{BC}$ \\
    \hline
    Real & & & & & & \\
    Ideal & & & & & &
\end{tabular}

\textbf{7. Determineu els coeficients de funcionament del cicle ideal. Com el cicle ideal d’Otto invers té les etapes de compressió i expansió reversibles i adiabàtiques, és tracta de trobar els estats finals d’aquestes etapes (estats 2 i 4), considerant que els estats 1 i 3 del cicle ideal són els mateixos que els del cicle real i que els volums dels estats 2 i 4 són també els mateixos (només canvien les pressions d’aquests dos darrers estats, que s’han de trobar suposant que les etapes de compressió i expansió són reversibles i adiabàtiques). Compareu i discutiu els valors dels COPs dels cicles ideal i real.}

\textbf{8. Representeu en un mateix diagrama pressió-volum, els cicles real i ideal. Compareu-los i discutiu les diferències}

A la \autoref{fig:cicle_otto_invers} es pot observar com hi ha lleugeres diferències entre el cicle d'Otto real i el cicle ideal. Aquestes diferències bàsicament són degudes a que el cilindre utilitzat per realitzar l'experiment no és completament adiabàtic i a l'hora de realitzar la compressió i la expansió hi havia intercanvi de calor a través de les parets del cilindre. Això provoca que en el cas de l'expansió el resultat obtingut sigui una pressió (degut a una major temperatura) lleugerament superior a l'esperada, el mateix passa pel cas de compressió.

\begin{figure}[H]
    \centering
    \caption{Cicle d'Otto invers}
    \label{fig:cicle_otto_invers}
    \begin{tikzpicture}[scale=1]
    \begin{axis}[
    xlabel = Volum ($cm^3$),
    ylabel = Pressió ($kPa$),
    ymajorgrids=true]
    \addplot[
    only marks, 
    green, 
    mark=halfcircle*, 
    mark size = 1pt, 
    opacity=0.5] 
    table [y=P, x=V, col sep=semicolon]{otto_invers.csv};
    \addlegendentry{Cicle d'Otto real};
    \addplot[
    only marks, 
    red, 
    mark=halfcircle*, 
    mark size = 1pt, 
    opacity=0.5] 
    table [y=P2, x=V, col sep=semicolon]{otto_invers.csv};
    \addlegendentry{Cicle d'Otto ideal};
    \end{axis}
    \end{tikzpicture}
\end{figure}

\subsection*{Cicle d'Otto invers modificat}

\textbf{9. Amb les dades experimentals completeu la següent taula:}

\begin{tabular}{c|cccc}
    & Estat 1 & Estat 2 & Estat 3 & Estat 4 \\
    \hline
    $P(kPa)$ & 226,645 & 230,956 & 90,865 & 102,18 \\
    $V(cm^3)$ & 92,174 & 92,174 & 198,527 & 198,527 \\
    $T(K)$ & 302,973 & 308,234 & 269,298 & 293,852 \\
\end{tabular}

\textbf{10. Determineu els treballs i les calors intercanviats a les tres etapes del cicle utilitzant les equacions (7), (8), (3) i (4). Determineu els coeficients de funcionament del cicle com a màquina frigorífica i com a bomba de calor utilitzant les calors i les equacions (6). Discutiu els resultats obtinguts.}

\textbf{11. Determineu els coeficients de funcionament del cicle modificat ideal. Per fer-ho cal que tracteu, com en el cicle d'Otto ideal, l'etapa d'expansió com si fos adiabàtica reversible. Compareu i discutiu els valors dels COPs dels cicles ideal i real.}

\begin{tabular}{c|cccccc}
    Cicle & $Q_H(J)$ & $Q_L(J)$ & $W_{exp}(J)$ & $W_{comp}(J)$ & $COP_{MF}$ & $COP_{BC}$ \\
    \hline
    Real & & & & & & \\
    Ideal & & & & & &
\end{tabular}

\textbf{12. Representeu en un mateix diagrama pressió-volum, els cicles real i ideal. Compareu-los i discutiu les diferències.}

\subsection*{Comparació de cicles}

\textbf{13. Compareu de forma gràfica o analítica els dos cicles estudiats (Otto i Otto modificat) i discutiu les diferències obtingudes en els valors dels treballs, calors i COPs}

\end{document}