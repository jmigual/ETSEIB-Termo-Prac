\documentclass[a4paper]{article}

\usepackage[catalan]{babel} % Language 
\usepackage{fontspec}
\usepackage[margin=2cm]{geometry}
\usepackage{pgfplots}
\usepackage{pgfplotstable}

\setlength{\parindent}{0pt}
\setlength{\parskip}{1em}

\title{\textsc{Laboratòri de Termodinàmica} \\ \textsc{Pràctica 2} \\ Compressió de gasos a baixes pressions \\
    \large
    Professor: Jose Luís \\ Grup: 11 }
\author{Joan Marcè Igual \and Esteve Tarragó Sanchís}
\date{26-9-2016}

\begin{document}
\maketitle

L'objectiu d'aquesta pràctica ha estat el de poder estudiar un gas ideal en dieferents processos de compressió. Per aquests processos s'ha determinat el coeficient politròpic $n$. 


\subsection*{Processos politròpics}

S'han realitzat dos assajos un en un procés adiabàtic (prova 1) i l'altre en un procés isotèrmic (prova 2). Per a cada assaig s'han obtingut les dades i també s'ha obtingut la recta d'interpolació d'aquestes dades ($ax^b$), a partir d'aquest exponent $b$ s'ha obtingut el coeficient politròpic (on $n = -b$).

\begin{tabular}{c|ccccccccc}
    Prova & $V_1$ $(cm^3)$ & $T_1$ $(K)$ & $P_1$ $(kPA)$ & $V_2$ $(cm^3)$ & $T_2$ $(K)$ & $P_2$ $(kPA)$ & $n$ & $W(J)$ & $Q(J)$ \\
    \hline
    1 & 198,53 & 300 & 103,258 & 90,336 & 381 & 381,604 & 1,29 & & \\
    2 & 198,692 & 300 & 102,719 & 90,498 & 302 & 228,262 & 1 & &  
\end{tabular}

En el següent gràfic (Pressió - Volum) s'han representat les dades obtingudes en els dos assajos anteriors i també s'hi ha afegit la funció d'interpolació usada.

\begin{center}
\begin{tikzpicture}[scale=1]
\begin{axis}[
    title = Assaig adiabàtic i isotèrmic,
    xlabel = Volum ($cm^3$),
    ylabel = Pressió ($kPa$),
    ymajorgrids=true]
\addplot[
    only marks, 
    green, 
    mark=halfcircle*, 
    mark size=1pt, 
    opacity=0.5] 
    table [y=P, x=V, col sep=semicolon]{politropics_adiabatic.csv};
\addlegendentry{Procés adiabàtic};
\addplot[green, domain = 90:200] {96988*x^-1.292};
\addlegendentry{$96988x^{-1.29}$};
\addplot[
    only marks, 
    red,  
    mark=halfcircle*, 
    mark size=1pt, 
    opacity=0.5] 
    table [y=P, x=V, col sep=semicolon]{politropics_isotermic.csv};
\addlegendentry{Procés isotèrmic};
\addplot[red, domain = 90:200] {20748*x^-1};
\addlegendentry{$20748x^{-1}$};
\end{axis}
\end{tikzpicture}
\end{center}

\subsection*{Cicle d'Otto invers}

\begin{center}
\begin{tikzpicture}[scale=1]
\begin{axis}[
    title = Cicle d'Otto invers,
    xlabel = Volum ($cm^3$),
    ylabel = Pressió ($kPa$),
    ymajorgrids=true]
\addplot[
    only marks, 
    green, 
    mark=halfcircle*, 
    mark size = 1pt, 
    opacity=0.5] 
    table [y=P, x=V, col sep=semicolon]{otto_invers.csv};
\addlegendentry{Cicle d'Otto gas real};
\addplot[
    only marks, 
    red, 
    mark=halfcircle*, 
    mark size = 1pt, 
    opacity=0.5] 
    table [y=P2, x=V, col sep=semicolon]{otto_invers.csv};
\addlegendentry{Cicle d'Otto gas ideal};
\end{axis}
\end{tikzpicture}
\end{center}

\begin{tabular}{c|cccc}
    & Estat 1 & Estat 2 & Estat 3 & Estat 4 \\
    \hline
    $P(kPa)$ & & & & \\
    $V(cm^3)$ & & & & \\
    $T(K)$ & & & & \\
\end{tabular}

\begin{tabular}{c|cccccc}
    Cicle & $Q_H(J)$ & $Q_L(J)$ & $W_{exp}(J)$ & $W_{comp}(J)$ & $COP_{MF}$ & $COP_{BC}$ \\
    \hline
    Real & & & & & & \\
    Ideal & & & & & &
\end{tabular}

\subsection*{Cicle d'Otto invers modificat}

\begin{tabular}{c|cccc}
    & Estat 1 & Estat 2 & Estat 3 & Estat 4 \\
    \hline
    $P(kPa)$ & & & & \\
    $V(cm^3)$ & & & & \\
    $T(K)$ & & & & \\
\end{tabular}

\begin{tabular}{c|cccccc}
    Cicle & $Q_H(J)$ & $Q_L(J)$ & $W_{exp}(J)$ & $W_{comp}(J)$ & $COP_{MF}$ & $COP_{BC}$ \\
    \hline
    Real & & & & & & \\
    Ideal & & & & & &
\end{tabular}

\end{document}